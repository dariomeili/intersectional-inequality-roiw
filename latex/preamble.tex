% Options for packages loaded elsewhere

\PassOptionsToPackage{hyphens}{url}
\PassOptionsToPackage{dvipsnames,svgnames*,x11names*}{xcolor}
%
\documentclass[12pt, a4paper]{article}

% Set up fonts

\usepackage{amssymb,amsmath} % Need to load before unicode-math


% Customize floats: always put captions at the top and use the
% afore-defined typeface in tables. This packages also provides the
% `\floatfoot' environment for notes in floats.
\usepackage{floatrow}
\floatsetup[table]{capposition=top, font={small}} % neet this for table caption to go on top
% Make float numbers and labels stand out
\usepackage[labelfont=bf]{caption}

% appendix titles
\usepackage[title]{appendix}

\usepackage[]{microtype}
\UseMicrotypeSet[protrusion]{basicmath} % disable protrusion for tt fonts
\usepackage{xcolor}
\usepackage{xurl} % add URL line breaks
\usepackage{bookmark}
\hypersetup{
  pdftitle={Intersectional Inequality in Education},
  pdfauthor={Dario Meili, Isabel Günther, Kenneth Harttgen},
  pdfkeywords={Inequality, Intersectionality, Measurement, Poverty},
  colorlinks=true,
  linkcolor=Maroon,
  filecolor=Maroon,
  citecolor=Blue,
  urlcolor=Blue,
  pdfcreator={LaTeX via pandoc}}
\urlstyle{same} % disable monospaced font for URLs
\usepackage{geometry}
\geometry{a4paper,
 left=25mm,
 top=25mm}
\usepackage{longtable,booktabs,dcolumn}

\usepackage{calc} % for calculating minipage widths
% Correct order of tables after \paragraph or \subparagraph
\usepackage{etoolbox}
\makeatletter
\patchcmd\longtable{\par}{\if@noskipsec\mbox{}\fi\par}{}{}
\makeatother
% Allow footnotes in longtable head/foot
\usepackage{footnotehyper}
\makesavenoteenv{longtable}
\usepackage{graphicx,subcaption}
\makeatletter
\def\maxwidth{\ifdim\Gin@nat@width>\linewidth\linewidth\else\Gin@nat@width\fi}
\def\maxheight{\ifdim\Gin@nat@height>\textheight\textheight\else\Gin@nat@height\fi}
\makeatother
% Scale images if necessary so that they will not overflow the page
% margins by default, and it is still possible to overwrite the defaults
% using explicit options in \includegraphics[width, height, ...]{}
\setkeys{Gin}{width=\maxwidth,height=\maxheight,keepaspectratio}
\setlength{\emergencystretch}{3em} % prevent overfull lines
\providecommand{\tightlist}{%
  \setlength{\itemsep}{0pt}\setlength{\parskip}{0pt}}
\setcounter{secnumdepth}{3}
\usepackage{float}
\usepackage{placeins}
\usepackage{array}
\usepackage{multirow}
\usepackage{wrapfig}
\usepackage{colortbl}
\usepackage{pdflscape}
\usepackage{tabu}
\usepackage{threeparttable}
\usepackage{threeparttablex}
\usepackage[normalem]{ulem}
%degree symbol
\usepackage{gensymb}
\usepackage{makecell}
\usepackage{siunitx} 
% needed for modelsummary
\usepackage{comment} % needed for commenting out sections
%bibliography
\usepackage{natbib}
\usepackage{setspace}%needed for onehlafspacing
\newcolumntype{d}{S[input-symbols = ()]} 
%needed for modelsummary



\newlength{\cslhangindent}
\setlength{\cslhangindent}{1.5em}
\newlength{\csllabelwidth}
\setlength{\csllabelwidth}{3em}
\newenvironment{CSLReferences}[2] % #1 hanging-ident, #2 entry spacing
 {% don't indent paragraphs
  \setlength{\parindent}{0pt}
  % turn on hanging indent if param 1 is 1
  \ifodd #1 \everypar{\setlength{\hangindent}{\cslhangindent}}\ignorespaces\fi
  % set entry spacing
  \ifnum #2 > 0
  \setlength{\parskip}{#2\baselineskip}
  \fi
 }%
 {}
 
\usepackage{calc}
\newcommand{\CSLBlock}[1]{#1\hfill\break}
\newcommand{\CSLLeftMargin}[1]{\parbox[t]{\csllabelwidth}{#1}}
\newcommand{\CSLRightInline}[1]{\parbox[t]{\linewidth - \csllabelwidth}{#1}\break}
\newcommand{\CSLIndent}[1]{\hspace{\cslhangindent}#1}

\usepackage{quotes}

%make it possible to control margins of landscape tables 
\makeatletter

\def\vfudge#1#2{%
\addtolength\textheight{#1}%
\@colroom\textheight
\vsize\textheight
\@colht\textheight
\def\LS@rot{%
  \setbox\@outputbox\vbox{\hbox{\kern-#2\rotatebox{90}{\box\@outputbox}}}}%
\clearpage}

\makeatother

% tabular array: for tinytable
\usepackage{tabularray}
\usepackage{float}
\usepackage{graphicx}
\usepackage{rotating}
\usepackage[normalem]{ulem}
\UseTblrLibrary{booktabs}
\UseTblrLibrary{siunitx}
\newcommand{\tinytableTabularrayUnderline}[1]{\underline{#1}}
\newcommand{\tinytableTabularrayStrikeout}[1]{\sout{#1}}
\NewTableCommand{\tinytableDefineColor}[3]{\definecolor{#1}{#2}{#3}}
 \noindent Intersectional inequality --- the notion that disparities run along combinations of social groups such as gender or ethnicity --- has become an increasingly prominent concept in the social sciences. However, there is little empirical research applying an intersectional framework to measure inequality. We propose two novel metrics of intersectional inequality based on the concept of horizontal inequality. Based on these measures, we analyze educational intersectionality in gender and ethnicity using Demographic and Health Survey data from 39 low- and middle-income countries and census data from the United States. We show that the intersectional perspective unveils substantial inequality that remains masked if gender and ethnicity are analyzed in isolation. For countries with high intersectional inequality, which is more than the sum of gender and ethnic inequality, reducing inequalities based on gender and ethnicity separately might not be enough to "leave no one behind," as the 2030 United Nations Agenda envisions.  \\
  
  \noindent \textbf{JEL codes}: D63, I24, J15, J16  \\
  \noindent \textbf{Keywords}: Inequality, Intersectionality, Measurement, Education  \\
